\documentclass[12pt,a4paper]{article}
\usepackage[utf8]{inputenc}
\usepackage[T1]{fontenc}
\usepackage{amsmath}
\usepackage{amsfonts}
\usepackage{amssymb}
\usepackage{amsthm}
\usepackage{graphicx}
\usepackage{subfigure}
\usepackage{float}
\newtheorem*{lemma}{Lemma}
\newtheorem*{theorem}{Theorem}
\newtheorem*{prf}{\textbf{Proof}}
\usepackage{caption}
\usepackage{graphicx}
\begin{document}
	\textbf{Assumptions:}
	\newline
	$\left(\mathbf{A}_{1}\right)$ Assume that $\exists B>0$ s.t. for all $x \in \mathcal{X}$,
	$\|k(x, .)\|_{\mathcal{H}_{0}} \leq B$ and $\left\|\nabla_{x} k(x, .)\right\|_{\mathcal{H}}=\left(\sum_{i=1}^{d}\left\|\partial_{x_{i}} k\left(x_{i}, .\right)\right\|_{\mathcal{H}_{0}}^{2}\right)^{\frac{1}{2}} \leq B$
	\newline
	$\left(\mathbf{A}_{2}\right)$The Hessian $H_{V}$ of $V=-\log \pi$ is well-defined and $\exists M>0$ s.t. $\left\|H_{V}(x)\right\|_{o p} \leq L_0+L_1||\nabla V(x)||$, for any $x\in\mathbb{R}^n$.
	\newline
	$\left(\mathbf{A}_{3}\right)$ Assume that $\exists$ is $C>0$ s.t. $\int ||\nabla V||d\mu_n\leq C$ for all $n$.
	\newline
	\newline
	Assumption(2) is weaker than the original $"L_0"-smoothness$, but assumption(3) is stronger than the original $"I_{stein}(\mu_n\mid \pi)\leq C"$ assumption. \textbf{If $\nabla V(x)\lesssim |x|^p$, we can delete assumption (3) and use $T_p-$inequality to analyze this like in the paper "Complexity Analysis of Stein Variational Gradient Descent Under Talagrand's Inequality T1". This maybe the spotlight of this idea, since we can deal with polynomial growth function instead of the original L-smoothness function(only quadratic growth)}.
	\newline
	\textbf{Proposition.} Assume that Assumptions $\left(\mathbf{A}_{1}\right)$ to $\left(\mathbf{A}_{3}\right)$ hold. Let $\alpha>1$ and choose $\gamma \leq \frac{\alpha-1}{\alpha B I_{stein}(\mu_n\mid \pi)^{\frac{1}{2}}}$.($I_{stein}(\mu_n\mid \pi)\leq B(\int ||\nabla V(x)||d\mu_n+1)$)Then:
	$$
	\mathrm{KL}\left(\mu_{n+1} \mid \pi\right)-\mathrm{KL}\left(\mu_{n} \mid \pi\right) \leq-\gamma\left(1-\gamma \frac{\left(\alpha^{2}+M\right) B^{2}}{2}\right) I_{s t e i n}\left(\mu_{n} \mid \pi\right),
	$$
	where $M= (\alpha^2 +L_{0}+L_{1}\left(\frac{L_{0}}{L_{1}} c+C\right) \exp (c))$
	\begin{proof}
		the only difference is to deal with $\psi_2(t):=\int\left.\left\langle g(x), H_{V}\left(\phi_{t}(x)\right) g(x)\right\rangle\right] d\mu_{n}(x) $.
		\begin{equation}
			\psi_2(t)\leq B^2I_{stein}(\mu_n\mid \pi)\int ||H_V(\phi_t(x))||_{op}
		\end{equation}
		\begin{equation}
			\begin{aligned}
				\left\| H_V (\phi(t))\right\|_{op} & \leq L_{0}+L_{1}\|\nabla V(\phi(t))\| \\
				& \leq L_{0}+L_{1}\left(\frac{L_{0}}{L_{1}} c t+\left\|\nabla V\left(\phi(0)\right)\right\|\right) \exp (c t),
			\end{aligned}
		\end{equation}
	where the first inequality in (2) is due to$ (\mathbf{A}_{2})$, the second inequality in (2) is due to the lemma below, $\phi(t)=I-tg, g=P_{\mu_{n}} \nabla \log \left(\frac{\mu_{n}}{\pi}\right)$, and $\|g(x)\| \leq B I_{\text {Stein }}\left(\mu_{n} \mid \pi\right)^{\frac{1}{2}}$. So finally we get
	\begin{equation}
		\psi_2(t)\leq (L_{0}+L_{1}\left(\frac{L_{0}}{L_{1}} c t+C\right) \exp (c t))B^2I_{stein}(\mu_n\mid \pi),
	\end{equation}
	combine with $\psi_1(t)\leq \alpha^2B^2I_{stein}(\mu_n\mid \pi)$, we get 
	\begin{equation}
		\varphi^{\prime \prime}(t)=\psi_{1}(t)+\psi_{2}(t)\leq (\alpha^2 +L_{0}+L_{1}\left(\frac{L_{0}}{L_{1}} c t+C\right) \exp (c t))B^2I_{stein}(\mu_n\mid \pi)
	\end{equation}
	\end{proof}

\begin{lemma}
	Let $V$ be $\left(L_{0}, L_{1}\right)$-smooth(satisfies assumption(2)), and $c>0$ be a constant. Given $x$, for any $x^{+}$such that $\left\|x^{+}-x\right\| \leq c / L_{1}$, we have $\left\|\nabla V\left(x^{+}\right)\right\| \leq e^{c}\left(\frac{c L_{0}}{L_{1}}+\|\nabla V(x)\|\right)$
\end{lemma} 
\begin{proof} Let $\gamma(t)$ be defined as $\gamma(t)=t\left(x^{+}-x\right)+x, t \in[0,1]$, then we have
$$
\nabla V(\gamma(t))=\int_{0}^{t} \nabla^{2} V(\gamma(\tau))\left(x^{+}-x\right) \mathrm{d} \tau+\nabla V(\gamma(0))
$$
We then bound the norm of $\nabla V(\gamma(t))$ :
$$
\begin{aligned}
	\|\nabla V(\gamma(t))\| & \leq \int_{0}^{t}\left\|\nabla^{2} V(\gamma(\tau))\left(x^{+}-x\right)\right\| \mathrm{d} \tau+\|\nabla V(\gamma(0))\| \\
	& \leq\left\|x^{+}-x\right\| \int_{0}^{t}\left\|\nabla^{2} V(\gamma(\tau))\right\| \mathrm{d} \tau+\|\nabla V(x)\| \\
	& \leq \frac{c}{L_{1}} \int_{0}^{t}\left(L_{0}+L_{1}\|\nabla V(\gamma(\tau))\|\right) \mathrm{d} \tau+\|\nabla V(x)\|
\end{aligned}
$$
The first inequality uses the triangular inequality of 2 -norm; The second inequality uses the property of spectral norm; The third inequality uses the definition of $\left(L_{0}, L_{1}\right)$-smoothness. By applying the Gronwall's inequality we get
$$
\|\nabla V(\gamma(t))\| \leq\left(\frac{L_{0}}{L_{1}} c t+\|\nabla V(x)\|\right) \exp (c t)
$$
The Lemma follows by setting $t=1$.
\end{proof}
\begin{lemma}
	$||x^{+}-x||=||x-\gamma g -x||\leq \gamma B I_{stein}(\mu_n\mid \pi)^{1/2}\leq \gamma B^2(\int |\nabla V(x)|d\mu_n(x)+1)$, so $c\leq (\gamma B^2(\int |\nabla V(x)|d\mu_n(x)+1))L_1$.(roughly this is an upper bound for c)
\end{lemma}

\end{document}